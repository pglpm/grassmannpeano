%\pdfoutput=1
%% Author: PGL  Porta Mana
%% Created: 2017-04-14T14:56:09+0200
%% Last-Updated: 2017-12-30T21:38:54+0100
%%%%%%%%%%%%%%%%%%%%%%%%%%%%%%%%%%%%%%%%%%%%%%%%%%%%%%%%%%%%%%%%%%%%%%
%Report-no: ***
\newcommand*{\memfontfamily}{zplx}
\newcommand*{\memfontpack}{newpxtext}
\documentclass[10pt,%extrafontsizes,%
%dvips,% uncomment this for arXiv
onecolumn,oneside,a5paper,article,frenchb,italian,german,swedish,latin,british%
]{memoir}
\newif\ifnotnotes
\notnotestrue % true = for publication, false = personal notes
\newcommand*{\pdftitle}{The beauty of Gra{\ss}mann spaces}
\newcommand*{\headtitle}{The beauty of Gra{\ss}mann spaces}
\newcommand*{\firstdraft}{18 April 2017}
\newcommand*{\pdfauthor}{I. Bengtsson, P.G.L.  Porta Mana}
\newcommand*{\headauthor}{\ifnotnotes Bengtsson \& Porta Mana%
\else\autanet\ Luca\fi}
\newcommand*{\reporthead}{}

%\usepackage{pifont}
%\usepackage{fontawesome}
\usepackage[T1]{fontenc} 
\input{glyphtounicode} \pdfgentounicode=1
\usepackage[utf8]{inputenx}
\usepackage{newunicodechar}
% \newunicodechar{Ĕ}{\u{E}}
% \newunicodechar{ĕ}{\u{e}}
% \newunicodechar{Ĭ}{\u{I}}
% \newunicodechar{ĭ}{\u{\i}}
% \newunicodechar{Ŏ}{\u{O}}
% \newunicodechar{ŏ}{\u{o}}
% \newunicodechar{Ŭ}{\u{U}}
% \newunicodechar{ŭ}{\u{u}}
% \newunicodechar{Ā}{\=A}
% \newunicodechar{ā}{\=a}
% \newunicodechar{Ē}{\=E}
% \newunicodechar{ē}{\=e}
% \newunicodechar{Ī}{\=I}
% \newunicodechar{ī}{\={\i}}
% \newunicodechar{Ō}{\=O}
% \newunicodechar{ō}{\=o}
% \newunicodechar{Ū}{\=U}
% \newunicodechar{ū}{\=u}
% \newunicodechar{Ȳ}{\=Y}
% \newunicodechar{ȳ}{\=y}

%\newcommand*{\bmmax}{0} % reduce number of bold fonts, before bm
\newcommand*{\hmmax}{0} % reduce number of heavy fonts, before bm
\usepackage{textcomp}
\usepackage{amsmath}
\usepackage{mathtools}
\setlength{\multlinegap}{0pt}

\usepackage{amssymb}
\usepackage{amsxtra}

\usepackage[british]{babel}\selectlanguage{british}
\newcommand*{\langfrench}{\foreignlanguage{french}}
\newcommand*{\langgerman}{\foreignlanguage{german}}
\newcommand*{\langitalian}{\foreignlanguage{italian}}
\newcommand*{\langswedish}{\foreignlanguage{swedish}}
\newcommand*{\langlatin}{\foreignlanguage{latin}}
\newcommand*{\langnohyph}{\foreignlanguage{nohyphenation}}

\usepackage[autostyle=false,autopunct=false,english=british]{csquotes}
\setquotestyle{american}
\renewcommand*{\mktextelpins}[1]{[\textellipsis\unkern\ #1]}
\renewcommand*{\mktextinselp}[1]{[#1 \textellipsis\unkern]}

\usepackage{amsthm}
\newcommand*{\QED}{\textsc{q.e.d.}}
\renewcommand*{\qedsymbol}{\QED}
\theoremstyle{remark}
\newtheorem{note}{Note}
\newtheorem*{remark}{Note}
\newtheoremstyle{innote}{\parsep}{\parsep}{\footnotesize}{}{}{}{0pt}{}
\theoremstyle{innote}
\newtheorem*{innote}{}

\usepackage[shortlabels,inline]{enumitem}
\SetEnumitemKey{para}{itemindent=\parindent,leftmargin=0pt,listparindent=\parindent,parsep=0pt,itemsep=\topsep}
% \begin{asparaenum} = \begin{enumerate}[para]
% \begin{inparaenum} = \begin{enumerate*}
\setlist[enumerate,2]{label=\alph*.}
\setlist[enumerate]{leftmargin=\parindent}
\setlist[itemize]{leftmargin=\parindent}
\setlist[description]{leftmargin=\parindent}

\usepackage[babel,theoremfont]{newpxtext}
\usepackage[bigdelims,nosymbolsc,smallerops]{newpxmath}
\useosf\linespread{1.083}


%% With euler font cursive for Greek letters - the [1] means 100% scaling
\DeclareFontFamily{U}{egreek}{\skewchar\font'177}%
\DeclareFontShape{U}{egreek}{m}{n}{<-6>s*[1]eurm5 <6-8>s*[1]eurm7 <8->s*[1]eurm10}{}%
\DeclareFontShape{U}{egreek}{m}{it}{<->s*[1]eurmo10}{}%
\DeclareFontShape{U}{egreek}{b}{n}{<-6>s*[1]eurb5 <6-8>s*[1]eurb7 <8->s*[1]eurb10}{}%
\DeclareFontShape{U}{egreek}{b}{it}{<->s*[1]eurbo10}{}%
\DeclareSymbolFont{egreeki}{U}{egreek}{m}{it}%
\SetSymbolFont{egreeki}{bold}{U}{egreek}{b}{it}% from the amsfonts package
\DeclareSymbolFont{egreekr}{U}{egreek}{m}{n}%
\SetSymbolFont{egreekr}{bold}{U}{egreek}{b}{n}% from the amsfonts package
% Take also \sum, \prod, \coprod symbols from Euler fonts
\DeclareFontFamily{U}{egreekx}{\skewchar\font'177}
\DeclareFontShape{U}{egreekx}{m}{n}{%
       <-7.5>s*[0.9]euex7%
    <7.5-8.5>s*[0.9]euex8%
    <8.5-9.5>s*[0.9]euex9%
    <9.5->s*[0.9]euex10%
}{}
\DeclareSymbolFont{egreekx}{U}{egreekx}{m}{n}
\DeclareMathSymbol{\sumop}{\mathop}{egreekx}{"50}
\DeclareMathSymbol{\prodop}{\mathop}{egreekx}{"51}
\DeclareMathSymbol{\coprodop}{\mathop}{egreekx}{"60}
\makeatletter
\def\sum{\DOTSI\sumop\slimits@}
\def\prod{\DOTSI\prodop\slimits@}
\def\coprod{\DOTSI\coprodop\slimits@}
\makeatother
% Greek letters not usually given in LaTeX. Comment the unneeded ones
\DeclareMathSymbol{\varpartial}{\mathalpha}{egreeki}{"40}
\DeclareMathSymbol{\partialup}{\mathalpha}{egreekr}{"40}
\DeclareMathSymbol{\alpha}{\mathalpha}{egreeki}{"0B}
\DeclareMathSymbol{\beta}{\mathalpha}{egreeki}{"0C}
\DeclareMathSymbol{\gamma}{\mathalpha}{egreeki}{"0D}
\DeclareMathSymbol{\delta}{\mathalpha}{egreeki}{"0E}
\DeclareMathSymbol{\epsilon}{\mathalpha}{egreeki}{"0F}
\DeclareMathSymbol{\zeta}{\mathalpha}{egreeki}{"10}
\DeclareMathSymbol{\eta}{\mathalpha}{egreeki}{"11}
\DeclareMathSymbol{\theta}{\mathalpha}{egreeki}{"12}
\DeclareMathSymbol{\iota}{\mathalpha}{egreeki}{"13}
\DeclareMathSymbol{\kappa}{\mathalpha}{egreeki}{"14}
\DeclareMathSymbol{\lambda}{\mathalpha}{egreeki}{"15}
\DeclareMathSymbol{\mu}{\mathalpha}{egreeki}{"16}
\DeclareMathSymbol{\nu}{\mathalpha}{egreeki}{"17}
\DeclareMathSymbol{\xi}{\mathalpha}{egreeki}{"18}
\DeclareMathSymbol{\omicron}{\mathalpha}{egreeki}{"6F}
\DeclareMathSymbol{\pi}{\mathalpha}{egreeki}{"19}
\DeclareMathSymbol{\rho}{\mathalpha}{egreeki}{"1A}
\DeclareMathSymbol{\sigma}{\mathalpha}{egreeki}{"1B}
\DeclareMathSymbol{\tau}{\mathalpha}{egreeki}{"1C}
\DeclareMathSymbol{\upsilon}{\mathalpha}{egreeki}{"1D}
\DeclareMathSymbol{\phi}{\mathalpha}{egreeki}{"1E}
\DeclareMathSymbol{\chi}{\mathalpha}{egreeki}{"1F}
\DeclareMathSymbol{\psi}{\mathalpha}{egreeki}{"20}
\DeclareMathSymbol{\omega}{\mathalpha}{egreeki}{"21}
\DeclareMathSymbol{\varepsilon}{\mathalpha}{egreeki}{"22}
\DeclareMathSymbol{\vartheta}{\mathalpha}{egreeki}{"23}
\DeclareMathSymbol{\varpi}{\mathalpha}{egreeki}{"24}
\let\varrho\rho 
\let\varsigma\sigma
\let\varkappa\kappa
\DeclareMathSymbol{\varphi}{\mathalpha}{egreeki}{"27}
%
\DeclareMathSymbol{\varAlpha}{\mathalpha}{egreeki}{"41}
\DeclareMathSymbol{\varBeta}{\mathalpha}{egreeki}{"42}
\DeclareMathSymbol{\varGamma}{\mathalpha}{egreeki}{"00}
\DeclareMathSymbol{\varDelta}{\mathalpha}{egreeki}{"01}
\DeclareMathSymbol{\varEpsilon}{\mathalpha}{egreeki}{"45}
\DeclareMathSymbol{\varZeta}{\mathalpha}{egreeki}{"5A}
\DeclareMathSymbol{\varEta}{\mathalpha}{egreeki}{"48}
\DeclareMathSymbol{\varTheta}{\mathalpha}{egreeki}{"02}
\DeclareMathSymbol{\varIota}{\mathalpha}{egreeki}{"49}
\DeclareMathSymbol{\varKappa}{\mathalpha}{egreeki}{"4B}
\DeclareMathSymbol{\varLambda}{\mathalpha}{egreeki}{"03}
\DeclareMathSymbol{\varMu}{\mathalpha}{egreeki}{"4D}
\DeclareMathSymbol{\varNu}{\mathalpha}{egreeki}{"4E}
\DeclareMathSymbol{\varXi}{\mathalpha}{egreeki}{"04}
\DeclareMathSymbol{\varOmicron}{\mathalpha}{egreeki}{"4F}
\DeclareMathSymbol{\varPi}{\mathalpha}{egreeki}{"05}
\DeclareMathSymbol{\varRho}{\mathalpha}{egreeki}{"50}
\DeclareMathSymbol{\varSigma}{\mathalpha}{egreeki}{"06}
\DeclareMathSymbol{\varTau}{\mathalpha}{egreeki}{"54}
\DeclareMathSymbol{\varUpsilon}{\mathalpha}{egreeki}{"07}
\DeclareMathSymbol{\varPhi}{\mathalpha}{egreeki}{"08}
\DeclareMathSymbol{\varChi}{\mathalpha}{egreeki}{"58}
\DeclareMathSymbol{\varPsi}{\mathalpha}{egreeki}{"09}
\DeclareMathSymbol{\varOmega}{\mathalpha}{egreeki}{"0A} 
%
\DeclareMathSymbol{\Alpha}{\mathalpha}{egreekr}{"41}
\DeclareMathSymbol{\Beta}{\mathalpha}{egreekr}{"42}
\DeclareMathSymbol{\Gamma}{\mathalpha}{egreekr}{"00}
\DeclareMathSymbol{\Delta}{\mathalpha}{egreekr}{"01}
\DeclareMathSymbol{\Epsilon}{\mathalpha}{egreekr}{"45}
\DeclareMathSymbol{\Zeta}{\mathalpha}{egreekr}{"5A}
\DeclareMathSymbol{\Eta}{\mathalpha}{egreekr}{"48}
\DeclareMathSymbol{\Theta}{\mathalpha}{egreekr}{"02}
\DeclareMathSymbol{\Iota}{\mathalpha}{egreekr}{"49}
\DeclareMathSymbol{\Kappa}{\mathalpha}{egreekr}{"4B}
\DeclareMathSymbol{\Lambda}{\mathalpha}{egreekr}{"03}
\DeclareMathSymbol{\Mu}{\mathalpha}{egreekr}{"4D}
\DeclareMathSymbol{\Nu}{\mathalpha}{egreekr}{"4E}
\DeclareMathSymbol{\Xi}{\mathalpha}{egreekr}{"04}
\DeclareMathSymbol{\Omicron}{\mathalpha}{egreekr}{"4F}
\DeclareMathSymbol{\Pi}{\mathalpha}{egreekr}{"05}
\DeclareMathSymbol{\Rho}{\mathalpha}{egreekr}{"50}
\DeclareMathSymbol{\Sigma}{\mathalpha}{egreekr}{"06}
\DeclareMathSymbol{\Tau}{\mathalpha}{egreekr}{"54}
\DeclareMathSymbol{\Upsilon}{\mathalpha}{egreekr}{"07}
\DeclareMathSymbol{\Phi}{\mathalpha}{egreekr}{"08}
\DeclareMathSymbol{\Chi}{\mathalpha}{egreekr}{"58}
\DeclareMathSymbol{\Psi}{\mathalpha}{egreekr}{"09}
\DeclareMathSymbol{\Omega}{\mathalpha}{egreekr}{"0A}
%
\DeclareMathSymbol{\alphaup}{\mathalpha}{egreekr}{"0B}
\DeclareMathSymbol{\betaup}{\mathalpha}{egreekr}{"0C}
\DeclareMathSymbol{\gammaup}{\mathalpha}{egreekr}{"0D}
\DeclareMathSymbol{\deltaup}{\mathalpha}{egreekr}{"0E}
\DeclareMathSymbol{\epsilonup}{\mathalpha}{egreekr}{"0F}
\DeclareMathSymbol{\zetaup}{\mathalpha}{egreekr}{"10}
\DeclareMathSymbol{\etaup}{\mathalpha}{egreekr}{"11}
\DeclareMathSymbol{\thetaup}{\mathalpha}{egreekr}{"12}
\DeclareMathSymbol{\iotaup}{\mathalpha}{egreekr}{"13}
\DeclareMathSymbol{\kappaup}{\mathalpha}{egreekr}{"14}
\DeclareMathSymbol{\lambdaup}{\mathalpha}{egreekr}{"15}
\DeclareMathSymbol{\muup}{\mathalpha}{egreekr}{"16}
\DeclareMathSymbol{\nuup}{\mathalpha}{egreekr}{"17}
\DeclareMathSymbol{\xiup}{\mathalpha}{egreekr}{"18}
\DeclareMathSymbol{\omicronup}{\mathalpha}{egreekr}{"6F}
 \DeclareMathSymbol{\piup}{\mathalpha}{egreekr}{"19}
\DeclareMathSymbol{\rhoup}{\mathalpha}{egreekr}{"1A}
\DeclareMathSymbol{\sigmaup}{\mathalpha}{egreekr}{"1B}
\DeclareMathSymbol{\tauup}{\mathalpha}{egreekr}{"1C}
\DeclareMathSymbol{\upsilonup}{\mathalpha}{egreekr}{"1D}
\DeclareMathSymbol{\phiup}{\mathalpha}{egreekr}{"1E}
\DeclareMathSymbol{\chiup}{\mathalpha}{egreekr}{"1F}
\DeclareMathSymbol{\psiup}{\mathalpha}{egreekr}{"20}
\DeclareMathSymbol{\omegaup}{\mathalpha}{egreekr}{"21}
\DeclareMathSymbol{\varepsilonup}{\mathalpha}{egreekr}{"22}
\DeclareMathSymbol{\varthetaup}{\mathalpha}{egreekr}{"23}
\DeclareMathSymbol{\varpiup}{\mathalpha}{egreekr}{"24}
\let\varrhoup\rhoup 
\let\varsigmaup\sigmaup
\let\varkappaup\kappaup
\DeclareMathSymbol{\varphiup}{\mathalpha}{egreekr}{"27}

% Optima as sans-serif font
\usepackage%[scaled=0.9]%
{classico}
\DeclareMathAlphabet{\mathsf}  {T1}{\sfdefault}{m}{sl}
\SetMathAlphabet{\mathsf}{bold}{T1}{\sfdefault}{b}{sl}
\newcommand*{\mathte}[1]{\textbf{\textit{\textsf{#1}}}}
% Upright sans-serif math alphabet
% \DeclareMathAlphabet{\mathsu}  {T1}{\sfdefault}{m}{n}
% \SetMathAlphabet{\mathsu}{bold}{T1}{\sfdefault}{b}{n}

% DejaVu Mono as typewriter text
\usepackage[scaled=0.84]{DejaVuSansMono}

\usepackage{mathdots}

\usepackage[usenames]{xcolor}
\definecolor{myblue}{RGB}{51,34,136}
\definecolor{mygreen}{RGB}{17,119,51}
\definecolor{myred}{RGB}{136,34,85}
\definecolor{myyellow}{RGB}{153,153,51}
\definecolor{mylightyellow}{RGB}{221,204,119}

\usepackage{bm}
\usepackage{microtype}

\usepackage[backend=biber,mcite,subentry,citestyle=numeric-comp,bibstyle=numericbringhurst,autopunct=false,sorting=none,sortcites=false,natbib=false,maxnames=8,minnames=8,giveninits=true,block=space,hyperref=true,defernumbers=false,useprefix=true,language=british]{biblatex}
\renewcommand*{\finalnamedelim}{, }
\setcounter{biburlnumpenalty}{1}
\setcounter{biburlucpenalty}{0}
\setcounter{biburllcpenalty}{1}
\DeclareDelimFormat{multicitedelim}{\addsemicolon\space}
\DeclareDelimFormat{postnotedelim}{\space}
\addbibresource{portamanabib.bib}
\renewcommand{\bibfont}{\footnotesize}
\defbibheading{bibliography}[\bibname]{\section*{#1}\addcontentsline{toc}{section}{#1}%\markboth{#1}{#1}
}
\newcommand*{\citep}{\parencites}
\newcommand*{\citey}{\parencites*}
\renewcommand*{\cite}{\citep}
\providecommand{\href}[2]{#2}
\providecommand{\eprint}[2]{\texttt{\href{#1}{#2}}}
\newcommand*{\amp}{\&}

\def\arxivp{}
\def\mparcp{}
\def\philscip{}
\def\biorxivp{}
\newcommand*{\arxivsi}{\texttt{arXiv} eprints available at \url{http://arxiv.org/}.\\}
\newcommand*{\mparcsi}{\texttt{mp\_arc} eprints available at \url{http://www.ma.utexas.edu/mp_arc/}.\\}
\newcommand*{\philscisi}{\texttt{philsci} eprints available at \url{http://philsci-archive.pitt.edu/}.\\}
\newcommand*{\biorxivsi}{\texttt{bioRxiv} eprints available at \url{http://biorxiv.org/}.\\}
\newcommand*{\arxiveprint}[1]{\global\def\arxivp{\arxivsi}%\citeauthor{0arxivcite}\addtocategory{ifarchcit}{0arxivcite}%eprint
\texttt{\urlalt{http://arxiv.org/abs/#1}{arXiv:\hspace{0pt}#1}}%
%\texttt{\href{http://arxiv.org/abs/#1}{\protect\url{arXiv:#1}}}%
%\renewcommand{\arxivnote}{\texttt{arXiv} eprints available at \url{http://arxiv.org/}.}
}
\newcommand*{\mparceprint}[1]{\global\def\mparcp{\mparcsi}%\citeauthor{0mparccite}\addtocategory{ifarchcit}{0mparccite}%eprint
\texttt{\urlalt{http://www.ma.utexas.edu/mp_arc-bin/mpa?yn=#1}{mp\_arc:\hspace{0pt}#1}}%
%\texttt{\href{http://www.ma.utexas.edu/mp_arc-bin/mpa?yn=#1}{\protect\url{mp_arc:#1}}}%
%\providecommand{\mparcnote}{\texttt{mp_arc} eprints available at \url{http://www.ma.utexas.edu/mp_arc/}.}
}
\newcommand*{\philscieprint}[1]{\global\def\philscip{\philscisi}%\citeauthor{0philscicite}\addtocategory{ifarchcit}{0philscicite}%eprint
\texttt{\urlalt{http://philsci-archive.pitt.edu/archive/#1}{PhilSci:\hspace{0pt}#1}}%
%\texttt{\href{http://philsci-archive.pitt.edu/archive/#1}{\protect\url{PhilSci:#1}}}%
%\providecommand{\mparcnote}{\texttt{philsci} eprints available at \url{http://philsci-archive.pitt.edu/}.}
}
\newcommand*{\biorxiveprint}[1]{\global\def\biorxivp{\biorxivsi}%\citeauthor{0arxivcite}\addtocategory{ifarchcit}{0arxivcite}%eprint
\texttt{\urlalt{http://biorxiv.org/content/early/#1}{bioRxiv:\hspace{0pt}#1}}%
%\texttt{\href{http://arxiv.org/abs/#1}{\protect\url{arXiv:#1}}}%
%\renewcommand{\arxivnote}{\texttt{arXiv} eprints available at \url{http://arxiv.org/}.}
}

\usepackage{graphicx}

\usepackage{hyperref}
\usepackage[depth=4]{bookmark}
\hypersetup{colorlinks=true,bookmarksnumbered,pdfborder={0 0 0.25},citebordercolor={0.2 0.1333 0.5333},%bluish
citecolor=myblue,linkbordercolor={0.0667 0.4667 0.2},%greenish
linkcolor=myred,urlbordercolor={0.5333 0.1333 0.3333},%reddish
urlcolor=mygreen,breaklinks=true,pdftitle={\pdftitle},pdfauthor={\pdfauthor}}

% \usepackage[vertfit=local]{breakurl}% only for arXiv
\providecommand*{\urlalt}{\href}

%%% Layout. I do not know on which kind of paper the reader will print the
%%% paper on (A4? letter? one-sided? double-sided?). So I choose A5, which
%%% provides a good layout for reading on screen and save paper if printed
%%% two pages per sheet. Average length line is 66 characters and page
%%% numbers are centred.
%\setstocksize{297mm}{210mm}%{*}% A4
\setstocksize{210mm}{5.5in}%{*}% 210x139.7
\settrimmedsize{\stockheight}{\stockwidth}{*}
\setlxvchars[\normalfont] %313.3632pt for a 66-characters line
\setxlvchars[\normalfont]
\setlength{\trimtop}{0pt}
\setlength{\trimedge}{\stockwidth}
\addtolength{\trimedge}{-\paperwidth}
%\settypeblocksize{*}{34pc}{1.618} % A4
\settypeblocksize{*}{26pc}{1.618}% nearer to a 66-line newpx and preserves GR
\setulmargins{*}{*}{1}%gives equal margins
\setlrmargins{*}{*}{*}
\setheadfoot{\onelineskip}{2\onelineskip}
\setheaderspaces{*}{*}{1}
\setmarginnotes{2ex}{10mm}{0pt}
\checkandfixthelayout[nearest]
\fixpdflayout
%%% End layout

%%% Sectioning
\newcommand*{\asudedication}[1]{%
{\par\centering\textit{#1}\par}}
\newenvironment{acknowledgements}{\section*{Acknowledgements}\addcontentsline{toc}{section}{Acknowledgements}}{\par}
\counterwithout{section}{chapter}
\setsecnumformat{\upshape\csname the#1\endcsname\quad}
\setsecheadstyle{\bfseries\sffamily\scshape\raggedright}
\setsubsecheadstyle{\bfseries\sffamily\raggedright}
\setsubsubsecheadstyle{\sffamily\slshape\raggedright}
\setparaheadstyle{\sffamily\slshape\raggedright}
\setcounter{secnumdepth}{2}
\setlength{\headwidth}{\textwidth}
\newcommand{\addchap}[1]{\chapter*[#1]{#1}\addcontentsline{toc}{chapter}{#1}}
\newcommand{\addsec}[1]{\section*{#1}\addcontentsline{toc}{section}{#1}}
\newcommand{\addsubsec}[1]{\subsection*{#1}\addcontentsline{toc}{subsection}{#1}}
\newcommand{\addsubsubsec}[1]{\subsubsection*{#1}\addcontentsline{toc}{subsubsection}{#1}}

% Headers and footers
\copypagestyle{manaart}{plain}
\makeheadrule{manaart}{\headwidth}{0.5\normalrulethickness}
\makeoddhead{manaart}{%
{\scriptsize\sffamily\scshape\headauthor}}{}{{\footnotesize\sffamily\headtitle}}
\makeoddfoot{manaart}{}{\thepage}{}
\makeatletter\newcommand\autanet{\includegraphics[height=\heightof{M}]{autanet.pdf}}
\makeatletter
\definecolor{mygray}{gray}{0.333}
%\newcommand\addprintnote{}% A4
\newcommand\addprintnote{\begin{picture}(0,0)%
\put(176,132){\makebox(0,0){\rotatebox{90}{\tiny\color{mygray}\textsf{This
            document is typographically designed for screen reading and
            two-up printing on A4 or Letter paper}}}}%
\end{picture}}
\makeoddfoot{plain}{}{\makebox[0pt]{\thepage}\addprintnote}{}
\makeoddhead{plain}{}{}{\footnotesize\reporthead}

\copypagestyle{manainitial}{plain}
\makeheadrule{manainitial}{\headwidth}{0.5\normalrulethickness}
\makeoddhead{manainitial}{%
\scriptsize\sffamily\scshape\headauthor}{}{\footnotesize\sffamily\headtitle}
\makeoddfoot{manaart}{}{\thepage}{}

\pagestyle{manaart}

\setlength{\droptitle}{-3.9\onelineskip}
\pretitle{\begin{center}\LARGE\sffamily\bfseries}
\posttitle{\bigskip\end{center}}

\makeatletter\newcommand\atf{\includegraphics[height=\heightof{t},trim=0pt 1pt 0pt 0pt]{at.pdf}}
\providecommand{\affiliation}[1]{\textsl{\textsf{\footnotesize #1}}}
\providecommand{\epost}[1]{\textnormal{\texttt{\footnotesize\textless#1\textgreater}}}
\providecommand{\email}[2]{\href{mailto:#1ZZ@#2 ((remove ZZ))}{#1\protect\atf#2}}

\preauthor{\vspace{-0\jot}\begin{center}
\normalsize\sffamily\lineskip  0.5em}
\postauthor{\par\end{center}}
\predate{\bigskip\begin{center}\footnotesize}
\postdate{\end{center}}
\usepackage{datetime2}
\DTMnewdatestyle{mydate}%
{% definitions
\renewcommand*{\DTMdisplaydate}[4]{%
\number##3\ \DTMenglishmonthname{##2} ##1}%
\renewcommand*{\DTMDisplaydate}{\DTMdisplaydate}%
}
\DTMsetdatestyle{mydate}


\setfloatadjustment{figure}{\footnotesize\centering}
\captiondelim{\quad}
\captionnamefont{\footnotesize\sffamily}
\captiontitlefont{\footnotesize}
\firmlists*
\midsloppy

% handling orphan/widow lines:
\clubpenalty=10000
\widowpenalty=10000
\raggedbottom

\selectlanguage{british}\frenchspacing
%%%% Paper's details %%%%
\title{\pdftitle%***
}
\author{\ifnotnotes% 
%  \hspace*{\stretch{1}}
  \parbox[t]{0.5\linewidth}%
{\protect\centering I.\ Bengtsson\\
  \epost{\email{ingemar}{fysik.su.se}}\\
    \affiliation{Stockholm University}}%
%  \hspace*{\stretch{1}}
  \parbox[t]{0.5\linewidth}%
{\protect\centering P.G.L.\ Porta\,Mana\\
  \epost{\email{pgl}{portamana.org}}\\
\affiliation{University of Cagliari}}%
%\hspace*{\stretch{1}}%
%\\[2\jot]%
%\affiliation{INM-6, Forschungszentrum Jülich, Germany}
% \\[2\jot]%
\else Luca\fi
%\quad
%\epost{\email{pgl}{portamana.org}}%
\\ {\footnotesize\textnormal{\emph{line art by I. Bengtsson}}}
}

\date{Draft of \today\ (first drafted \firstdraft)}

%@@@@@@@@@@@@@@@@@@@@@@@ new commands @@@@@@@@@@@@@@@@@@@@@
% \providecommand{\nequiv}{\not\equiv}
% \providecommand{\coloneqq}{\mathrel{\mathop:}=}
% \providecommand{\eqqcolon}{=\mathrel{\mathop:}}
% \providecommand{\varprod}{\prod}
\newcommand*{\de}{\partialup}%partial diff
% \newcommand*{\pu}{\piup}%constant pi
% \newcommand*{\delt}{\deltaup}%Kronecker, Dirac
% \newcommand*{\eps}{\varepsilonup}%Levi-Civita, Heaviside
% \newcommand*{\riem}{\zetaup}%Riemann zeta
% \providecommand{\degree}{\textdegree}% degree
% \newcommand*{\celsius}{\textcelsius}% degree Celsius
% \newcommand*{\micro}{\textmu}% degree Celsius
% \newcommand*{\I}{\mathrm{i}}%imaginary unit
% \newcommand*{\e}{\mathrm{e}}%Neper
\newcommand*{\di}{\mathrm{d}}%differential
% \newcommand*{\Di}{\mathrm{D}}%capital differential
% \newcommand*{\planckc}{\hslash}
% \newcommand*{\avogn}{N_{\textrm{A}}}
% \newcommand*{\NN}{\bm{\mathrm{N}}}
% \newcommand*{\ZZ}{\bm{\mathrm{Z}}}
% \newcommand*{\QQ}{\bm{\mathrm{Q}}}
\newcommand*{\RR}{\bm{\mathrm{R}}}
% \newcommand*{\CC}{\bm{\mathrm{C}}}
% \newcommand*{\nabl}{\bm{\nabla}}%nabla
% \DeclareMathOperator{\lb}{lb}%base 2 log
% \DeclareMathOperator{\tr}{tr}%trace
% \DeclareMathOperator{\card}{card}%cardinality
% \DeclareMathOperator{\im}{Im}%im part
% \DeclareMathOperator{\re}{Re}%re part
\DeclareMathOperator{\sgn}{sgn}%signum
% \DeclareMathOperator{\ent}{ent}%integer less or equal to
% \DeclareMathOperator{\Ord}{O}%same order as
% \DeclareMathOperator{\ord}{o}%lower order than
% \newcommand*{\incr}{\triangle}%finite increment
\newcommand*{\defd}{\coloneqq}
\newcommand*{\defs}{\eqqcolon}
% \newcommand*{\Land}{\bigwedge}
% \newcommand*{\Lor}{\bigvee}
% \newcommand*{\lland}{\mathbin{\ \land\ }}
% \newcommand*{\llor}{\mathbin{\ \lor\ }}
% \newcommand*{\lonlyif}{\mathbin{\Rightarrow}}%implies
% \newcommand*{\limplies}{\mathbin{\Rightarrow}}%implies
% \newcommand*{\mimplies}{\Rightarrow}%implies
% \newcommand*{\liff}{\mathbin{\Leftrightarrow}}%if and only if
% \newcommand*{\cond}{\mathpunct{|}}%conditional sign (in probabilities)
% \newcommand*{\lcond}{\mathpunct{|\ }}%conditional sign (in probabilities)
% \newcommand*{\bigcond}{\mathpunct{\big|}}%conditional sign (in probabilities)
% \newcommand*{\lbigcond}{\mathpunct{\big|\ }}%conditional sign (in probabilities)
% \newcommand*{\suchthat}{\mid}%{\mathpunct{|}}%such that (eg in sets)
% \newcommand*{\bigst}{\mathpunct{\big|}}%such that (eg in sets)
% \newcommand*{\with}{\colon}%with (list of indices)
% \newcommand*{\mul}{\times}%multiplication
% \newcommand*{\inn}{\cdot}%inner product
% \newcommand*{\dotv}{\mathord{\,\cdot\,}}%variable place
% \newcommand*{\comp}{\circ}%composition of functions
% \newcommand*{\con}{\mathbin{:}}%scal prod of tensors
% \newcommand*{\equi}{\sim}%equivalent to 
% \newcommand*{\corr}{\mathrel{\hat{=}}}%corresponds to
% \providecommand{\varparallel}{\ensuremath{\mathbin{/\mkern-7mu/}}}%parallel (tentative symbol)
\renewcommand*{\le}{\leqslant}%less or equal
\renewcommand*{\ge}{\geqslant}%greater or equal
\DeclarePairedDelimiter\clcl{[}{]}
\DeclarePairedDelimiter\clop{[}{[}
\DeclarePairedDelimiter\opcl{]}{]}
\DeclarePairedDelimiter\opop{]}{[}
\DeclarePairedDelimiter\abs{\lvert}{\rvert}
% \DeclarePairedDelimiter\norm{\lVert}{\rVert}
\DeclarePairedDelimiter\set{\{}{\}}
% \DeclareMathOperator{\pr}{P}%probability
% \newcommand*{\pf}{\mathrm{p}}%probability
% \newcommand*{\p}{\mathrm{P}}%probability
% \newcommand*{\tf}{\mathrm{T}}%probability
% \renewcommand*{\|}{\cond}
% \newcommand*{\+}{\lor}
% \renewcommand{\*}{\land}
\newcommand*{\sect}{\S}% Sect.~
\newcommand*{\sects}{\S\S}% Sect.~
\newcommand*{\chap}{ch.}%
\newcommand*{\chaps}{chs}%
% \newcommand*{\fn}{fn}%
\newcommand*{\eqn}{eq.}%
\newcommand*{\eqns}{eqs}%
\newcommand*{\fig}{fig.}%
\newcommand*{\figs}{figs}%
\newcommand*{\vs}{{vs.}}
\newcommand*{\etc}{{etc.}}
\newcommand*{\ie}{{i.e.}}
% \newcommand*{\ca}{{c.}}
% \newcommand*{\Ie}{{I.e.}}
% \newcommand*{\Eg}{{E.g.}}
% \newcommand*{\eg}{{e.g.}}
% \newcommand*{\viz}{{viz}}
\newcommand*{\cf}{{cf.}}
% \newcommand*{\Cf}{{Cf.}}
% \newcommand*{\vd}{{v.}}
% \newcommand*{\Vd}{{V.}}
\newcommand*{\etal}{{et al.}}
% \newcommand*{\etsim}{{et sim.}}
% \newcommand*{\ibid}{{ibid.}}
% \newcommand*{\sic}{{sic}}
% \newcommand*{\id}{\mathte{I}}%id matrix
% \newcommand*{\bd}{\hspace{0pt}}%
% \newcommand*{\nbd}{\nobreakdash}%
% \def\hy{-\penalty0\hskip0pt\relax}
% \newcommand*{\labelbis}[1]{\tag*{(\ref{#1})$_\text{r}$}}
% \newcommand*{\mathbox}[2][.8]{\parbox[t]{#1\columnwidth}{#2}}
% \newcommand*{\zerob}[1]{\makebox[0pt][l]{#1}}
\newcommand*{\tprod}{\mathop{\textstyle\prod}\nolimits}
\newcommand*{\tsum}{\mathop{\textstyle\sum}\nolimits}
% \newcommand*{\tint}{\begingroup\textstyle\int\endgroup\nolimits}
% \newcommand*{\tland}{\mathop{\textstyle\bigwedge}\nolimits}
% \newcommand*{\tlor}{\mathop{\textstyle\bigvee}\nolimits}
% \newcommand*{\sprod}{\mathop{\textstyle\prod}}
% \newcommand*{\ssum}{\mathop{\textstyle\sum}}
% \newcommand*{\sint}{\begingroup\textstyle\int\endgroup}
% \newcommand*{\sland}{\mathop{\textstyle\bigwedge}}
% \newcommand*{\slor}{\mathop{\textstyle\bigvee}}
% \newcommand*{\T}{^\intercal}%transpose
% \newcommand*{\E}{\mathrm{E}}
% \DeclarePairedDelimiter\expp{(}{)}
% \newcommand*{\expe}{\E\expp}%round
% \newcommand*{\expeb}{\E\clcl}%square
% %\newcommand*{\QEM}%{\textnormal{$\Box$}}%{\ding{167}}
% \newcommand*{\qem}{\leavevmode\unskip\penalty9999 \hbox{}\nobreak\hfill
% \quad\hbox{\QEM}}
\definecolor{notecolour}{RGB}{68,170,153}
\newcommand*{\mynote}[1]{ {\color{notecolour}\maltese\ #1}}
%\newcommand*{\widebar}[1]{{\mkern1.5mu\skew{2}\overline{\mkern-1.5mu#1\mkern-1.5mu}\mkern 1.5mu}}

\newcommand*{\gm}{Graßmann}
\newcommand*{\ve}{\curlyvee}
\newcommand*{\we}{\curlywedge}
\newcommand*{\yr}{r}
\newcommand*{\yN}{N}
\newcommand*{\yw}{w}
\newcommand*{\ym}{m}
\newcommand*{\ya}{a}
\newcommand*{\yb}{b}
\newcommand*{\ye}{e}
\newcommand*{\yC}{C}
\newcommand*{\yal}{\alpha}
%@@@@@@@@@@@@@@@@@@@@@@@ new commands end @@@@@@@@@@@@@@@@@

\firmlists
\begin{document}
\captiondelim{\quad}\captionnamefont{\footnotesize}\captiontitlefont{\footnotesize}
\selectlanguage{british}\frenchspacing

%%% Title and abstract %%%
\maketitle
\ifnotnotes
\abstractrunin
\abslabeldelim{}
\renewcommand*{\abstractname}{}
\setlength{\absleftindent}{0pt}
\setlength{\absrightindent}{0pt}
\setlength{\abstitleskip}{-\absparindent}
\begin{abstract}\labelsep 0pt%
  \noindent{\textsf{Abstract:}}\quad\gm\ spaces are very beautiful indeed.
\par%\\[\jot]
\noindent
{\footnotesize PACS: ***}\qquad%
{\footnotesize MSC: ***}%
%\qquad{\footnotesize Keywords: ***}
\end{abstract}\fi

\selectlanguage{british}\frenchspacing
%\asudedication{to ***}
%\setlength{\epigraphwidth}{.7\columnwidth}
%\setlength{\epigraphrule}{0pt}
%\epigraph{}

{\footnotesize\mynote{IMPORTANT: several statements below are \emph{false}
    and will be corrected later. This happens because they  concern
    notions at the \enquote{boundaries} of the references given in the
    first section, which aren't developed in any of those works. For
    example, statements about the scalar multiplication of outer-weighted
    objects. I'm working this out.}}

\mynote{Most important todo:
  \begin{itemize}
  \item clarify if and how Whitney forms and \enquote{elements} fit into
    \gm\ algebra.
  \end{itemize}
}

\mynote{Other refs to check:  \cite{barnabeietal1985,crapo2009,brinietal2011,bossavit1999,whitney1957,bossavit2002_r2005,arnoldetal2006,sternetal2015}}

\section{Why}
\label{sec:why}

These notes present \gm\ spaces by combining \gm's ideas as summarized by
Peano \citey{peano1888}, ideas from geometric algebra
\cite{dorstetal2007,li2008}, the concept of outer-oriented or
\enquote{twisted} geometric objects
\cite{veblenetal1932,schoutenetal1940,schouten1951_r1989,burke1983,burke1985_r1987,burke1995,bossavit1994_r2002,bossavit2003b},
the theory of Whitney elements introduced by Bossavit
\cite{bossavit2002_r2005,bossavit2003b,whitney1957,gawliketal2010,brinietal2011,arnoldetal2009_r2010}
% de Rham cohomology \cite{derham1955_t1984,arnoldetal2009_r2010},
and insights from Barnabei, Brini, Rota's
\citey{brinietal2011,barnabeietal1985}, Goldman's
\citey{goldman2002,goldman2000}, and Crapo's \citey{crapo2009} brilliant
presentations. \mynote{add also
  \cite{bambergetal1988_r1990,bambergetal1990_r1992,frankel1997_r2012}}

These notes do not contain anything new, except maybe the glue joining the
works above. We will rest content if we manage to spark your curiosity and
to spur you to take a look at them.

\gm\ spaces are beautiful. They share many important properties with
projective spaces, yet are as intuitive as affine spaces. They represent
notions like point, vector, functional, form on an equal geometric level.
Their geometric operations have a beautiful algebra and include a
simplified version of the operations of of integration, differentiation,
boundary and the de~Rham theory of chains and cochains on manifolds. Their
geometric notions can be given intuitive physical interpretations and be
computationally used for the numerical solution of partial
integro-differential equations. \mynote{Maybe add something about their
  relation to Cayley algebras}

The primitives of \gm\ spaces are the same as for affine spaces: point,
straight line, plane, and so on; and the notion of parallelism. But each of
these geometric objects has also a \emph{weight density} and an
\emph{orientation}. Weight and orientation can be of \emph{inner} kind,
that is, defined on the geometric object itself; or of \emph{outer} kind,
that is, defined on the complement subspace of the object.

%\section{Primitives}
%\label{sec:primitives}

\section{Basic geometric objects and properties}
\label{sec:basic_objects_properties}


\subsection{Basic geometric objects}
\label{sec:points_etc}

The basic geometric objects of a \gm\ space are those of an affine space:
points, straight lines, planes, and so on. We assume these are well-known
to you. Let's agree to use the terms \emph{point}, \emph{line},
\emph{plane} with their usual $0$-, $1$-, $2$-dimensional meanings. We call
\emph{$\yr$-plane} their $\yr$-dimensional generalization, a $0$-plane
being a point, and so on. In a space of dimension $\yN$ we call
\emph{hyperplane} an $(\yN-1)$-plane; there is only one $\yN$-plane, the
space itself.

The notion of parallelism is very important, and we assume that it is also
well-known to you. Just like in an affine space, there is no notion of
angle or orthogonality, nor is there an absolute measure of length, area,
and so on. We want to remind that the notion of parallelism, though, allows
us to meaningfully speak of the ratio of lengths of segments lying on
parallel lines, the ratio of areas lying on parallel planes, and so on.

Important: a weight density, to be introduced shortly, is \emph{not} the
same as a relative length, area, \etc, although these notions are related.

\subsection{Complements}
\label{sec:complements}

The notion of \emph{complement} is essential. Let's explain it in detail,
starting with an example.

Consider a line in a 3-dimensional space. We can consider the set of this
line and all lines parallel to it, see \fig\mynote{}. The \enquote{points} of
this set are the parallel lines of our original space. This set is
2-dimensional and has an affine structure and a notion of parallelism. For
example, two parallel lines in this set correspond to two parallel planes
containing some of the parallel lines our set is made of. This set is
called the \emph{complement} of our line.

If we select a plane intersecting our line in only one point, then there is
a one-one correspondence between it and the complement. For example, a
point on the plane corresponds to the unique line parallel to our initial
one and passing through this point, and vice versa; and this line is a
\enquote{point} in the complement. We can therefore speak of objects,
properties, constructions on the complement of our line by referring to a
specific plane intersecting this line -- remembering, though, that such
properties are not specific to that plane but shared by all other planes
intersecting this line.

Generalizing to an $\yr$-plane in an $\yN$-dimensional space, its
complement is $(\yN-\yr)$-dimensional and is in one-one correspondence with
each $(\yN-\yr)$-plane intersecting the original $\yr$-plane in only one
point.

Two parallel $\yr$-planes have the same complement. The complement of a
point is isomorphic to the whole space, and the complement of the whole
space is in one-one correspondence with each point.


%\subsection{Weight densities and orientations, inner and outer}
%\label{sec:weights_orientations}


\subsection{Weight densities}
\label{sec:weight_densities}

With each $\yr$-plane we can associate a \emph{weight density}, or simply
\enquote{weight}, represented by a non-negative real number. Of a point we
simply say that it has a weight.

Assigning a weight to an $\yr$-plane is different from assigning a unit
length to it, although the two notions are related. Consider for example
the two parallel lines of \fig\mynote{}. The first has a total weight $\yw$
associated with the segment $AB$, the second a total weight $2\yw$
associated with the segment $CD$, which has twice the length of $AB$. The two
parallel lines have therefore the same weight density.

It is therefore only meaningful to compare the weight densities of parallel
$\yr$-planes.


\subsection{Orientations}
\label{sec:orientations}

It is easy to familiarize with the notion of orientation of a line, plane,
volume. We can imagine to traverse a line with two points $A$, $B$ by going
from $A$ to $B$ or from $B$ to $A$. On a plane we can imagine to
\enquote{spin} clockwise or counter-clockwise. In a volume we can identify
two cork-screw senses. In each case there are two possible orientations. In
the $0$-dimensional case of a point we convene to have the two symbolic
orientations \enquote{$+$} and \enquote{$-$}. 

Mathematically orientation is usually defined in terms of some equivalence
class. This shows that, as with all notions defined in terms of equivalence
classes, the mathematical formalism is still too primitive to capture it
well. In this note we would like to consider orientation as intuitive and
primitive, and will try not to use equivalence classes to define it.

The choice of an orientation on a line determines a unique ordering of
every two distinct points on it. The choice of an orientation on a plane
determines a unique ordering of every three non-collinear points on it,
modulo an even number of permutations in the ordering. And so on for
$\yr$-planes.

If a particular orientation is understood on a set of $\yr$-planes, we can
understand a negative weight as indicating a reversed orientation. This
corresponds to multiplying the positive weight by $-1$, as explained below.


\subsection{Inner and outer properties}
\label{sec:inner_outer_properties}

Weight and orientation were defined above as \enquote{lying} within an
$\yr$-plane. For this reason they are called \emph{inner}, and we say that
an $\yr$-plane has an inner weight or is inner-oriented.

Now consider the complement of an $\yr$-plane in an $\yN$-dimensional
space. This is an $(\yN-\yr)$-dimensional affine space, and also the unique
$(\yN-\yr)$-plane within this space. We can associate with it a weight and an
orientation. As it happens with all properties defined on a complement, they
can be mapped one-to-one onto every $(\yN-\yr)$-plane intersecting the
initial $\yr$-plane in our original space; each such $(\yN-\yr)$-plane
therefore acquires a weight and an orientation. Vice versa we can
speak of the weight and orientation of a complement by referring to any
$(\yN-\yr)$-plane intersecting our initial $\yr$-plane.

When we assign a weight $\yw$ to the complement of an $\yr$-plane, we say
that the latter has an \emph{outer} weight $1/\yw$ -- note the reciprocal.
When we assign an orientation to the complement, we say that the
$\yr$-plane has an \emph{outer} orientation.

Inner and outer weights and orientations can be assigned independently of
each other. Given an $\yr$-plane we have therefore four possibilities: we
can assign to it an inner weight and inner orientation, an outer weight and
outer orientation, an inner weight and outer orientation, an outer weight
and inner orientation. \mynote{Can we also assign all four?}

Let's see the special case of points in 2-dimensional space, \ie\ a plane;
see \fig\mynote{}. A point with inner weight and inner orientation simply
has an associated real number, positive or negative. A point with outer
weight $\ym$ and outer orientation assigns a weight surface density $1/\ym$
and a circulation sense to the whole plane. A point with inner weight and
outer orientation has an associated positive real number and a circulation
for the plane; a negative real number indicates that the opposite
circulation must be taken. Finally, a point with outer weight $\ym$ and
inner orientation assigns a weight surface density $1/\ym$ to the whole
plane and has an associated $+$ or $-$ sign; a negative density indicates
that the opposite sign must be taken.

The case of a line in 3-dimensional space is illustrated in \fig\mynote{}.


\section{Scalar multiplication and sum}
\label{sec:multiplication_sum}

We indicate weighted and oriented $\yr$-planes simply by their weights, if
this does not cause confusion. A negative weight means an opposite
orientation with respect to a tacitly understood one.

Each $\yr$-plane with a weight and orientation, either inner or outer, can
be multiplied by a real number. The result is the same $\yr$-plane with its
weight multiplied by the absolute value of the number, and the same or a
reversed orientation depending on whether the number is positive or
negative. See \fig\mynote{}

Two $\yr$-planes, both having inner or outer properties, can also be
summed. 

Let's start with the sum of weighted points, $\ym_1$, $\ym_2$, \ldots. This
is an interesting operation because it generalizes affine combination and
leads to properties alike those of projective space.

First assume that the sum of the weights does not vanish:
$\sum_j\ym_j \ne 0$. The result of the sum of the weighted points is a
point given by the usual affine combination of the points with normalized
coefficients $\ym_i/\sum_j\ym_j$, and with an associated weight
$\sum_j\ym_j$. \mynote{Add definition or example of affine combination}.
The resulting weight and orientation are inner or outer depending on
whether those of all the summand points are.

Now consider the case of vanishing total weight, and for simplicity
consider just two points: $\ym_1 + \ym_2 = 0$. By writing $\ym_2 = \epsilon
-\ym_1$ and considering smaller and smaller values of $\epsilon$, we see
that the resulting point is \enquote{at infinity} and has a vanishing
weight; its orientation is therefore also undetermined.

Points of this kind are called \emph{vectors}. They have indeed all
properties of usual \enquote{free vectors}. Consider for example two points
$A$, $B$ with unit weights and positive orientation, and the vector
$v \defd B-A$. By summing this vector to the point $A$ we obtain the point
$B$: $A+v = B$. Thus $v$ really behaves as a vector from $A$ to $B$,
translating the point $A$ to $B$ when summed to the former.

Vectors also give \gm\ spaces properties typical of projective spaces.
Consider for example the unit-weight points $A$, $B$ on the line $\ya$, and
two unit-weight points $C$, $D$ on the parallel line $\yb$ and having the
same distance as $A$, $B$. The vectors $B-A$ and $D-C$ are the same. This
can be seen by rewriting the equality $B-A = D-C$ as $A+D = B+C$, which is
indeed satisfied because the result of both $A+D$ and $B+C$ is the point at
the centre of their parallelogram, with a weight of $2$. This means that
the vector $B-A$ is a point belonging to both parallel lines $\ya$ and
$\yb$, which therefore \enquote{meet at infinity}. A vector therefore
characterizes the common direction, or \emph{attitude}, and inner
orientation of a set of parallel lines.

But the presence of weights leads to a much richer and interesting
structure than in affine and projective spaces. Consider for example the
vector $v \defd B-A$, and sum it to the point $A$ having a large weight
$\ym$, which we denote $\ym A$. The result is the point $B+(\ym-1)A$,
located at relative distances $(\ym-1)/\ym$ from $B$ and $1/\ym$ from $A$.
As the weight $\ym$ increases, this resultant point gets closer to $A$.
Thus the amount by which a weighted point is translated by a vector depends
on the point's weight, a \enquote{heavy} point being translated less than a
\enquote{light} one.

For an exploration of these fascinating properties and their use we
recommend Goldman's brilliant articles \cite{goldman2000,goldman2002}.

\mynote{Explore and add more curiosities of this kind}


\medskip

Above we spoke of the total sum of the weights of some points, implicitly
assuming that these weights can somehow be summed independently of the
points they belong to. This is possible because all points have the same
complement -- the whole space -- or equivalently because they are all
\enquote{parallel} to one another.

The weights of parallel $\yr$-planes can similarly be added, taking care of
their orientations, independently of the $\yr$-planes they belong to. The
reason is exactly the same as for the comparison of lengths, surfaces,
\etc, of parallel objects in an affine space.

Sums of parallel $\yr$-planes with total zero weight are objects called
$\yr$-vectors, analogous to vectors. They can be imagined as $\yr$-planes
\enquote{at infinity} with vanishing weight. An $\yr$-vector is
\enquote{common} to a set of parallel $(\yr+1)$-planes, and therefore
characterizes their $\yr$-direction or attitude and inner orientation.
Again this gives many projective-like properties to a \gm\ space.

\mynote{The following example needs the introduction of wedge first} Let's
consider the two-dimensional case of \emph{bivectors}. Consider the
antiparallel lines $A\we B$ and $-C\we D$, the distances $AB$ and $CD$
being equal. The total weight of these lines is zero. The sum
$A \we B - C \we D$ is a bivector. It can be considered as a
\enquote{line} that \enquote{goes around} the plane containing $A \we B$
and $C \we D$, with a circulation sense agreeing with those of $AB$ and
$CD$. See \fig\mynote{}.

The segment $CD$ is parallel to $AB$, hence there is a vector $w$ such that
$C = A + w$ and $D = B + w$. The bivector
$\yal \defd A \we B - C \we D$ is therefore equal to $(B-A) \we w$,
or $\yal = v \we w$, with $v \defd B-A$. A bivector is therefore equal to
the product of two vectors.

Very interesting is the case of $\yr$-planes that are skew, that is, do not
lie on a common $(\yr+1)$-plane. Their weights cannot meaningfully be
summed. ***


\mynote{Explain sum of lines and planes with examples}

The literature uses the term \emph{extensors} to indicate points, lines,
\etc\ that can be both of the usual kind and \enquote{at infinity},
reserving \enquote{point} \etc\ only for those not at infinity.
\mynote{Shall we also use it?}

\section{Join, meet, contraction}
\label{sec:join_meet_contraction}

Besides sum and scalar multiplication, \gm\ spaces have three kinds of
multiplicative operations: join, meet, and contraction.

The join  operates on inner-weighted objects, giving a new inner-weighted
object of higher dimension. The meet  operates on outer-weighted
objects, giving a new outer-weighted object of lower dimension, hence
higher codimension. 


\subsection{Wedge or join} 
\label{sec:wedge} 

The common notion of vector is usually associated with a line, but we have seen
that vectors in \gm\ spaces are special kinds of points, and indeed they
have properties more similar to points than lines.

A line with inner weight and orientation also has many similarities to a
vector. The main difference is that it is not a free vector, because it is
bound to that particular line; but it is not a bound vector either, because
it has no definite initial point within that line.

\mynote{Not sure whether the paragraphs above are understandable}

\gm\ spaces have also another operation, the \emph{wedge product}, which
yields \emph{inner-weighted} objects of higher dimension from
lower-dimensional ones.

Let's start with two inner-weighted, inner-oriented points $\ym_1$,
$\ym_2$. The result of their wedge product $\ym_1 \we \ym_2$ is the line
passing through them, having weight $\ym_1\ym_2$ and inner orientation
going from the first to the second; if the total weight is negative then
the opposite orientation must be taken. See \fig\mynote{}.


\mynote{Clarify the difference between \enquote{wedge} and \enquote{join}
  in the literature}

The wedge product of \emph{outer-weighted} objects yields a
lower-dimensional object instead.\mynote{That's why Barnabei, Brini, Rota
  \cite{barnabeietal1985,brinietal2011} indicate the wedge of
  inner-weighted objects with $\lor$ and that of outer-weighted ones with
  $\land$, reminding of $\cup$ and $\cap$.}


\medskip


Consider the exterior algebra of a 4-dimensional vector space, with four
linearly independent vectors $\ye_1$, $\ye_2$, $\ye_3$, $\ye_4$. The
products $\ye_1\we\ye_2$ and $\ye_3\we\ye_4$ can be associated with two
planes containing the respective vectors. These planes  intersect at one
point only, the origin. It is for this reason that the sum $\ye_1\we\ye_2
+ \ye_3\we\ye_4$ cannot be written as $a \we b$.

% This does not happen for the wedge product of a \gm\ space. Consider four
% non-coplanar points $A_1$, $A_2$, $A_3$, $A_4$. We have $A_1\we A_2 +
% A_2\we A_3 + A_3\we A_4 + A_4\we A_1 = 0$, which implies $A_1 \we
% A_2 + A_3 \we A_4 = A_1 \we A_4 + A_3 \we A_2$. Now consider the
% midpoints $B$ between $A_1$ and $A_3$,  $B \defd (A_1 + A_3)/2$, and $C$
% between $A_2$ and $A_4$, $C \defd (A_2 + A_4)/2$. Their product, given the
% equality just proven, is
% \begin{equation}
%   \label{eq:ABCDisEF}
%   \begin{split}
%     B \we C &= (A_1 \we A_2 + A_3\we A_4)/4 +
%     (A_1 \we A_4 + A_3\we A_2)/4,\\
%     &= (A_1 \we A_2 + A_3\we A_4)/2.
%   \end{split}
% \end{equation}
% A sum of weighted lines of the form $A_1 \we A_2 + A_3 \we A_4$ can therefore be
% interpreted as a weighted line: $2 B \we C$.



\section{Contraction or meet}
\label{sec:contraction}

\mynote{There are two main approaches around:
  \begin{itemize}
  \item The \enquote{geometric-algebra school} introduce a metric and a
    dual space, and define the meet in terms of contractions with duals and
    scalar products. Whitney seems to do likewise, at least at the
    beginning of his book.
  \item The \enquote{Rota school} introduce a volume element
    ($\yN$-covector) and define the meet in terms of the latter. Their
    approach is to view a \emph{Peano space} in terms of invariance under
    the special linear group -- that is, the preservation of volumes.
  \end{itemize}
  Neither school introduce a distinction between inner- and outer-weighted
  objects. Barnabei, Brini, \amp\ Rota make one valid observation:
  \begin{quote}\small
    Elie Cartan found the regressive product to be superfluous and awkward.
    By vector space duality, a pairing of the two exterior algebras of $V$
    and $V^*$ could easily be made with only one kind of product, the one
    that came to be called the wedge. \textelp{However, T}he dual space
    $V^*$ of a vector space $V$ plays no role in such a calculus: a
    hyperplane is an object living in $V$, and its identification with a
    linear functional is a step backwards in clarity.
  \end{quote}
  I think that their geometric viewpoint would be even more elegant by
  distinguishing between inner and outer weights. Covectors are like
  vectors, but characterized by an outer weight. From this point of view,
  moreover, we note that the traditional wedge of covectors yields their
  \emph{intersection}, whereas the wedge of vectors give their
  \emph{union}. This different behaviour is a direct consequence of the
  only ways in which inner and outer weights can be meaningfully be
  combined -- and it doesn't require the notion of contraction or a special
  $\yN$-covector; both may be introduced later. }



\section{Bases}
\label{sec:bases}

\bigskip

\section{Boundary and differential}
\label{sec:differential}
\mynote{The following sections would aim to embed the calculus of
  simplicial $\yr$-chains, presented in Bossavit \citey[esp.
  \sects~23.2--3]{bossavit2002_r2005} into the algebra of \gm\ spaces. It's
  not clear whether this is meaningful or possible yet, but some very
  interesting mathematical curiosities suggest that those two theories are
  part of one common framework.
  \\[\jot]
  One curiosity is that the boundary of an $\yr$-chain always seems to
  correspond to an $(\yr-1)$-vector, \ie\ an $(\yr-1)$-plane \enquote{at
    infinity} with vanishing weight. Consider for example the segment $AB$.
  Its boundary is $\de(AB) = B - A$, which interpreted within \gm-space
  algebra is a vector, a point at infinity with vanishing weight. Same goes
  for the triangle $ABC$: its boundary is $\de(ABC) = AB + BC + CA$. If we
  interpret juxtaposition as wedge product, we see that the result is a
  bivector, a line at infinity with vanishing weight. So one wonders
  whether the calculus of simplicial $\yr$-chains can be somehow
  interpreted as a sub-calculus of $\yr$-vectors in \gm\ space. If this is
  true, then what is the relation between the boundary operator $\de$ and
  the \gm\ operations?
  \\[\jot]
  A related question appears considering for example a $0$-chain like
  $\sum_i \ym_i A_i$. In the chain calculus this is just a formal
  expression; in \gm\ space it is equivalent to a specific point $B$ with a
  specific weight. Does this mean that, when we \enquote{integrate} a
  $0$-form on this chain, the result is the same as evaluating it on the
  resulting point $B$? It's indeed a possibility, given the affine
  structure in which both are embedded. }


In the theory of chains and differential forms on manifolds, a simplicial
$\yr$-chain is defined as a \emph{formal} sum $\sum_i \yw_i \yC_i$ of
simplicial $\yr$-submanifolds $\yC_i$ with weights $\yw_i$. Differential
$\yr$-forms can then be integrated on such chains.

In a \gm\ space we can interpret the formal sum above as a new, specific
$\yr$-plane $\yC'$ with an associated weight and, owing to the affine
properties of the space and its algebra, the \enquote{integral} of an
$\yr$-form on this $\yr$-plane is equal to its integral on the initial
chain. This is the basis for discretization and interpolation schemes to
numerically solve partial integro-differential equations on manifolds.

\mynote{Whitney forms = basis forms in the spaces of $\yr$-planes
 derived from $\yN+1$ basis points?}

\section{Whitney, de Rham, computation}
\label{sec:whitney_derham}


\section{Representing physical quantities}
\label{sec:physical_quantities}

\begin{quotation}
  Physical vector quantities may be divided into two classes, in one of
  which the quantity is defined with reference to a line, while in the
  other the quantity is defined with reference to an area. \textelp{}
  
  There is another distinction between different kinds of directed
  quantities \textelp{}. This is the distinction between longitudinal and
  rotational properties.\sourceatright{\emph{Maxwell \citey[\sects~13,
      15]{maxwell1873_r1881}}}
\end{quotation}

Physical quantities are associated with a point, like temperature; with a
line, like electromotive force; with a surface, like mass or energy flux or
current; with a volume, like charge. And we suppose them to vary with a
particular kind of continuity with respect to variations of these geometric
extensions. For this reason they are represented by inner- or
outer-oriented differential forms, defined on a manifold and meant to be
evaluated at a point or integrated over a line, surface, volume. The laws
that govern them are expressed as particular mathematical relations among
such integrals.

Approximate solutions to particular problems may be obtained by selecting a
discrete and finite set of points, lines, surfaces, volumes on the
manifold, requiring the laws to be satisfied within this set, and then
interpolating the values of the fields on geometric extensions outside this
set.

For example, we can associate a temperature with each point in such
discrete set, and an electromotive force with each line, a current with
each surface, a charge with each volume therein. A \gm\ space comes handy
in such a discretization: the amount of a physical quantity over a
particular point, line, and so on is represented by a single geometric
entity, like a weighted point and so on. The interpolation is then automatically
achieved by the sum of such geometric entities in the \gm\ space.



\iffalse
\begin{figure}[!b]
\centering
\includegraphics[width=0.4\columnwidth]{***}%
\caption{***}
\label{***}
\end{figure}
\fi

\bigskip
\ifnotnotes
\begin{acknowledgements}
  PGLPM thanks  Mari \amp\ Miri for continuous encouragement and
  affection, Buster Keaton for filling life with awe and inspiration, and
  the developers and maintainers of \LaTeX, Emacs, AUC\TeX, MiK\TeX,
  arXiv, biorXiv, PhilSci, Hal archives, Python, Inkscape, Sci-Hub for
  making a free and unfiltered scientific exchange possible.
%\rotatebox{15}{P}\rotatebox{5}{I}\rotatebox{-10}{P}\rotatebox{10}{\reflectbox{P}}\rotatebox{-5}{O}.
\end{acknowledgements}
\else\sourceatright{\autanet}\fi

%\appendixpage
%\appendix

%%%%%%%%%%%%%%% BIB %%%%%%%%%%%%%%%

\defbibnote{postnote}{\small\par\medskip\noindent{\footnotesize% Note:
\arxivp \mparcp \philscip \biorxivp}%
}

\newcommand{\citein}[2][]{\textnormal{\textcite[#1]{#2}}%\addtocategory{extras}{#2}
}
\newcommand{\citebi}[2][]{\cite[#1]{#2}%\addtocategory{extras}{#2}
}
\newcommand{\subtitleproc}[1]{}

\printbibliography[postnote=postnote]


\end{document}
---------- cut text ----------------


%%% Local Variables: 
%%% mode: LaTeX
%%% TeX-PDF-mode: t
%%% TeX-master: t
%%% End: 
